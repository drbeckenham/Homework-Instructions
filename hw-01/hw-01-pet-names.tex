\documentclass[]{tufte-handout}

% ams
\usepackage{amssymb,amsmath}

\usepackage{ifxetex,ifluatex}
\usepackage{fixltx2e} % provides \textsubscript
\ifnum 0\ifxetex 1\fi\ifluatex 1\fi=0 % if pdftex
  \usepackage[T1]{fontenc}
  \usepackage[utf8]{inputenc}
\else % if luatex or xelatex
  \makeatletter
  \@ifpackageloaded{fontspec}{}{\usepackage{fontspec}}
  \makeatother
  \defaultfontfeatures{Ligatures=TeX,Scale=MatchLowercase}
  \makeatletter
  \@ifpackageloaded{soul}{
     \renewcommand\allcapsspacing[1]{{\addfontfeature{LetterSpace=15}#1}}
     \renewcommand\smallcapsspacing[1]{{\addfontfeature{LetterSpace=10}#1}}
   }{}
  \makeatother

\fi

% graphix
\usepackage{graphicx}
\setkeys{Gin}{width=\linewidth,totalheight=\textheight,keepaspectratio}

% booktabs
\usepackage{booktabs}

% url
\usepackage{url}

% hyperref
\usepackage{hyperref}

% units.
\usepackage{units}


\setcounter{secnumdepth}{-1}

% citations
\usepackage{natbib}
\bibliographystyle{plainnat}


% pandoc syntax highlighting
\usepackage{color}
\usepackage{fancyvrb}
\newcommand{\VerbBar}{|}
\newcommand{\VERB}{\Verb[commandchars=\\\{\}]}
\DefineVerbatimEnvironment{Highlighting}{Verbatim}{commandchars=\\\{\}}
% Add ',fontsize=\small' for more characters per line
\newenvironment{Shaded}{}{}
\newcommand{\AlertTok}[1]{\textcolor[rgb]{1.00,0.00,0.00}{\textbf{#1}}}
\newcommand{\AnnotationTok}[1]{\textcolor[rgb]{0.38,0.63,0.69}{\textbf{\textit{#1}}}}
\newcommand{\AttributeTok}[1]{\textcolor[rgb]{0.49,0.56,0.16}{#1}}
\newcommand{\BaseNTok}[1]{\textcolor[rgb]{0.25,0.63,0.44}{#1}}
\newcommand{\BuiltInTok}[1]{\textcolor[rgb]{0.00,0.50,0.00}{#1}}
\newcommand{\CharTok}[1]{\textcolor[rgb]{0.25,0.44,0.63}{#1}}
\newcommand{\CommentTok}[1]{\textcolor[rgb]{0.38,0.63,0.69}{\textit{#1}}}
\newcommand{\CommentVarTok}[1]{\textcolor[rgb]{0.38,0.63,0.69}{\textbf{\textit{#1}}}}
\newcommand{\ConstantTok}[1]{\textcolor[rgb]{0.53,0.00,0.00}{#1}}
\newcommand{\ControlFlowTok}[1]{\textcolor[rgb]{0.00,0.44,0.13}{\textbf{#1}}}
\newcommand{\DataTypeTok}[1]{\textcolor[rgb]{0.56,0.13,0.00}{#1}}
\newcommand{\DecValTok}[1]{\textcolor[rgb]{0.25,0.63,0.44}{#1}}
\newcommand{\DocumentationTok}[1]{\textcolor[rgb]{0.73,0.13,0.13}{\textit{#1}}}
\newcommand{\ErrorTok}[1]{\textcolor[rgb]{1.00,0.00,0.00}{\textbf{#1}}}
\newcommand{\ExtensionTok}[1]{#1}
\newcommand{\FloatTok}[1]{\textcolor[rgb]{0.25,0.63,0.44}{#1}}
\newcommand{\FunctionTok}[1]{\textcolor[rgb]{0.02,0.16,0.49}{#1}}
\newcommand{\ImportTok}[1]{\textcolor[rgb]{0.00,0.50,0.00}{\textbf{#1}}}
\newcommand{\InformationTok}[1]{\textcolor[rgb]{0.38,0.63,0.69}{\textbf{\textit{#1}}}}
\newcommand{\KeywordTok}[1]{\textcolor[rgb]{0.00,0.44,0.13}{\textbf{#1}}}
\newcommand{\NormalTok}[1]{#1}
\newcommand{\OperatorTok}[1]{\textcolor[rgb]{0.40,0.40,0.40}{#1}}
\newcommand{\OtherTok}[1]{\textcolor[rgb]{0.00,0.44,0.13}{#1}}
\newcommand{\PreprocessorTok}[1]{\textcolor[rgb]{0.74,0.48,0.00}{#1}}
\newcommand{\RegionMarkerTok}[1]{#1}
\newcommand{\SpecialCharTok}[1]{\textcolor[rgb]{0.25,0.44,0.63}{#1}}
\newcommand{\SpecialStringTok}[1]{\textcolor[rgb]{0.73,0.40,0.53}{#1}}
\newcommand{\StringTok}[1]{\textcolor[rgb]{0.25,0.44,0.63}{#1}}
\newcommand{\VariableTok}[1]{\textcolor[rgb]{0.10,0.09,0.49}{#1}}
\newcommand{\VerbatimStringTok}[1]{\textcolor[rgb]{0.25,0.44,0.63}{#1}}
\newcommand{\WarningTok}[1]{\textcolor[rgb]{0.38,0.63,0.69}{\textbf{\textit{#1}}}}

% table with pandoc

% multiplecol
\usepackage{multicol}

% strikeout
\usepackage[normalem]{ulem}

% morefloats
\usepackage{morefloats}


% tightlist macro required by pandoc >= 1.14
\providecommand{\tightlist}{%
  \setlength{\itemsep}{0pt}\setlength{\parskip}{0pt}}

% title / author / date
\title{HW 01 - Pet names}
\author{Susan Beckenham}
\date{}


\begin{document}

\maketitle




\begin{marginfigure}
\includegraphics[width=0.8\linewidth]{img/jovana-askrabic-XYIQXLH_v0o-unsplash} \caption[Photo by Jovana Askrabic on Unsplash]{Photo by Jovana Askrabic on Unsplash}\label{fig:photo}
\end{marginfigure}

The goal of this assignment is to introduce you to R, RStudio, Git, and
GitHub, which you'll be using throughout the course both to learn the
data science concepts discussed in the course and to analyze real data
and come to informed conclusions.

\subsection{What You'll Learn}\label{what-youll-learn}

By the end of this assignment, you will be able to:

\begin{itemize}
\tightlist
\item
  Clone a GitHub repository to JupyterHub
\item
  Navigate the RStudio interface
\item
  Edit and knit R Markdown documents
\item
  Use basic R functions to explore data
\item
  Make commits with meaningful messages
\item
  Push changes to GitHub
\item
  Generate a PDF document for submission
\end{itemize}

\textbf{Don't worry if these terms are unfamiliar!} We'll walk through
each step carefully.

\begin{center}\rule{0.5\linewidth}{0.5pt}\end{center}

\section{Getting started}\label{getting-started}

\subsection{Prerequisites}\label{prerequisites}

This assignment assumes that you have reviewed the lectures titled
``Meet the toolkit: Programming'' and ``Meet the toolkit: version
control and collaboration''. If you haven't yet done so, please pause
and complete the following before continuing.

\subsection{Terminology}\label{terminology}

We've already thrown around a few new terms, so let's define them before
we proceed.

\begin{itemize}
\item
  \textbf{R:} Name of the programming language we will be using
  throughout the course.
\item
  \textbf{RStudio:} An integrated development environment for R. In
  other words, a convenient interface for writing and running R code.
\item
  \textbf{Git:} A version control system.
\item
  \textbf{GitHub:} A web platform for hosting version controlled files
  and facilitating collaboration among users.
\item
  \textbf{Repository:} A Git repository contains all of your project's
  files and stores each file's revision history. It's common to refer to
  a repository as a repo.

  \begin{itemize}
  \tightlist
  \item
    In this course, each assignment you work on will be contained in a
    Git repo.
  \item
    For individual assignments, only you will have access to the repo.
    For team assignments, all team members will have access to a single
    repo where they work collaboratively.
  \item
    All repos associated with this course are housed in the course
    GitHub organization. The organization is set up such that students
    can only see repos they have access to, but the course staff can see
    all of them.
  \end{itemize}
\end{itemize}

\subsection{Starting slow}\label{starting-slow}

As the course progresses, you are encouraged to explore beyond what the
assignments dictate; a willingness to experiment will make you a much
better programmer! Before we get to that stage, however, you need to
build some basic fluency in R. First, we will explore the fundamental
building blocks of all of these tools.

Before you can get started with the analysis, you need to make sure you:

\begin{itemize}
\item
  have a GitHub account
\item
  are a member of the course GitHub organization
\item
  have successfully logged in and authenticated in the JupyterHub
\end{itemize}

If you failed to confirm any of these, it means you have not yet
completed the prerequisites for this assignment. Please go back to
\hyperref[prerequisites]{Prerequisites} and complete them before
continuing the assignment.

\section{Workflow}\label{workflow}

\begin{marginfigure}
\textbf{IMPORTANT:} If there is no GitHub repo created for you for this
assignment, it means I didn't have your GitHub username as of when I
assigned the homework. Please let me know your GitHub username asap, and
I can create your repo.
\end{marginfigure}

For each assignment in this course you will start with a GitHub repo
that I created for you and that contains the starter documents you will
build upon when working on your assignment. The first step is always to
bring these files into RStudio so that you can edit them, run them, view
your results, and interpret them. This action is called
\textbf{cloning}.

Then you will work in RStudio on the data analysis, making
\textbf{commits} along the way (snapshots of your changes) and finally
\textbf{push} all your work back to GitHub.

The next few steps will walk you through the process of getting
information of the repo to be cloned, cloning your repo in a new RStudio
project, and getting started with the analysis.

\subsection{Step 1. Get URL of repo to be
cloned}\label{step-1.-get-url-of-repo-to-be-cloned}

\begin{marginfigure}
\includegraphics[width=0.8\linewidth]{img/clone-repo-link} \end{marginfigure}

On GitHub, click on the green \textbf{Code} button, select
\textbf{HTTPS} (this might already be selected by default, and if it is,
you'll see the text \emph{Use Git or checkout with SVN using the web
URL} as in the image on the right). Click on the clipboard icon 📋 to
copy the repo URL.

\subsection{Step 2. Go to JupyterHub and open
RStudio}\label{step-2.-go-to-jupyterhub-and-open-rstudio}

Go to \href{https://santiago.cloudbank.2i2c.cloud/hub/login}{JuptyerHub}
and then \textbf{open an RStudio Notebook}.

\begin{marginfigure}
\includegraphics[width=0.8\linewidth]{img/jupyterhub} \end{marginfigure}

\subsection{Step 3. Clone the repo}\label{step-3.-clone-the-repo}

In RStudio, click on the \textbf{down arrow} next to New Project and
then choose \textbf{New Project from Git Repository}.

In the pop-up window, \textbf{paste the URL} you copied from GitHub,
make sure the box for \textbf{Add packages from the base project} is
checked (it should be, by default) and then click \textbf{OK}.

\begin{flushleft}\includegraphics[width=0.8\linewidth]{img/new-project-from-git} \end{flushleft}

✓ Checkpoint: You should now see your project files in the Files pane
(bottom right). If you don't see a file called
\texttt{Homework\ Instructions}, something went wrong - ask for help
before continuing.

\section{Hello RStudio!}\label{hello-rstudio}

RStudio is comprised of four panes.

\begin{figure*}
\includegraphics[width=0.8\linewidth]{img/rstudio-anatomy} \end{figure*}

\begin{itemize}
\tightlist
\item
  On the bottom left is the Console, this is where you can write code
  that will be evaluated. Try typing \texttt{2\ +\ 2} here and hit
  enter, what do you get?
\item
  On the bottom right is the Files pane, as well as other panes that
  will come handy as we start our analysis.
\item
  If you click on a file, it will open in the editor, on the top left
  pane.
\item
  Finally, the top right pane shows your Environment. If you define a
  variable it would show up there. Try typing
  \texttt{x\ \textless{}-\ 2} in the Console and hit enter, what do you
  get in the \textbf{Environment} pane? Importantly, this pane is also
  where the \textbf{Git} interface lives. We will be using that
  regularly throughout this assignment.
\end{itemize}

\section{Warm up}\label{warm-up}

Before we introduce the data, let's warm up with some simple exercises.

\begin{marginfigure}
The top portion of your R Markdown file (between the three dashed lines)
is called \textbf{YAML}. It stands for ``YAML Ain't Markup Language''.
It is a human friendly data serialization standard for all programming
languages. All you need to know is that this area is called the YAML (we
will refer to it as such) and that it contains meta information about
your document.
\end{marginfigure}

\subsection{Step 1. Update the YAML}\label{step-1.-update-the-yaml}

Open the R Markdown (Rmd) file in your project, change the author name
to your name, and knit the document.

\begin{center}\includegraphics[width=0.8\linewidth]{img/yaml-raw-to-rendered} \end{center}

\subsection{Step 2: Commit}\label{step-2-commit}

Then Go to the \textbf{Git pane} in your RStudio.

You should see that your Rmd (R Markdown) file and its output, your md
file (Markdown), are listed there as recently changed files.

Next, click on \textbf{Diff}. This will pop open a new window that shows
you the \textbf{diff}erence between the last committed state of the
document and its current state that includes your changes. If you're
happy with these changes, click on the checkboxes of all files in the
list, and type \emph{``Update author name''} in the \textbf{Commit
message} box and hit \textbf{Commit}.

\begin{flushleft}\includegraphics[width=0.8\linewidth]{img/update-author-name-commit} \end{flushleft}

You don't have to commit after every change, this would get quite
cumbersome. You should consider committing states that are
\emph{meaningful to you} for inspection, comparison, or restoration. In
the first few assignments we will tell you exactly when to commit and in
some cases, what commit message to use. As the semester progresses we
will let you make these decisions.

\subsection{Step 3. Push}\label{step-3.-push}

Now that you have made an update and committed this change, it's time to
push these changes to the web! Or more specifically, to your repo on
GitHub. Why? So that others can see your changes. And by others, we mean
the course teaching team (your repos in this course are private to you
and us, only). In order to push your changes to GitHub, click on
\textbf{Push}.

\begin{marginfigure}
\includegraphics[width=0.8\linewidth]{img/ready-to-push} \end{marginfigure}

This will prompt a dialogue box where you may need to authenticate with
GitHub. Follow the prompts to complete the authentication process.

\textbf{Note:} The first time you push, you may need to set up
authentication. If you encounter issues, refer to the authentication
guide posted on Canvas or ask for help during office hours.

\textbf{Thought exercise:} Which of the above steps (updating the YAML,
committing, and pushing) needs to talk to GitHub?\footnote{Only pushing
  requires talking to GitHub, this is why you're asked for your password
  at that point.} Pushing needs to talk to Github.

\textbf{✓ Checkpoint:} After pushing, go to your GitHub repo in your web
browser and refresh the page. You should see your updated file with your
name in it. If you don't see the changes, try pushing again or ask for
help.

\section{Packages}\label{packages}

R is an open-source language, and developers contribute functionality to
R via packages. In this assignment we will use the following packages:

\begin{itemize}
\tightlist
\item
  \textbf{tidyverse}: a collection of packages for doing data analysis
  in a ``tidy'' way
\item
  \textbf{openintro}: a package that contains the datasets from
  OpenIntro resources
\end{itemize}

We use the \texttt{library()} function to load packages. In your R
Markdown document you should see an R chunk labelled
\texttt{load-packages} which has the necessary code for loading both
packages. You should also load these packages in your Console, which you
can do by sending the code to your Console by clicking on the
\textbf{Run Current Chunk} icon (green arrow pointing right icon).

\begin{flushleft}\includegraphics[width=0.8\linewidth]{img/load-packages-chunk} \end{flushleft}

Note that these packages also get loaded in your R Markdown environment
when you \textbf{Knit} your R Markdown document.

\textbf{✓ Checkpoint:} If the packages loaded successfully, you should
see no error messages in red. If you see an error like ``there is no
package called\ldots{}'', let your instructor know.

\section{Data}\label{data}

The city of \href{https://en.wikipedia.org/wiki/Seattle}{Seattle, WA}
has an open data portal that includes pets registered in the city. For
each registered pet, we have information on the pet's name and species.
The data used in this exercise can be found in the \textbf{openintro}
package, and it's called \texttt{seattlepets}. Since the dataset is
distributed with the package, we don't need to load it separately; it
becomes available to us when we load the package.

You can view the dataset as a spreadsheet using the \texttt{View()}
function. Note that you should not put this function in your R Markdown
document, but instead type it directly in the Console, as it pops open a
new window (and the concept of popping open a window in a static
document doesn't really make sense\ldots). When you run this in the
console, you'll see the following \textbf{data viewer} window pop up.

\begin{Shaded}
\begin{Highlighting}[]
\FunctionTok{View}\NormalTok{(seattlepets)}
\end{Highlighting}
\end{Shaded}

\begin{flushleft}\includegraphics[width=0.8\linewidth]{img/view-data} \end{flushleft}

You can find out more about the dataset by inspecting its documentation
(which contains a \textbf{data dictionary}, name of each variable and
its description), which you can access by running \texttt{?seattlepets}
in the Console or using the Help menu in RStudio to search for
\texttt{seattlepets}.

\subsection{Common Issues and
Solutions}\label{common-issues-and-solutions}

As you work through this assignment, you might encounter some issues.
Here are the most common ones:

\textbf{Knitting errors:}

\begin{itemize}
\tightlist
\item
  \textbf{Error: ``there is no package called\ldots{}''} → Run
  \texttt{library(tidyverse)} and \texttt{library(openintro)} in your
  Console first
\item
  \textbf{Error: ``object not found''} → Make sure you've run all code
  chunks before the one causing the error
\item
  \textbf{Can't find the Knit button} → Look at the top of the editor
  pane (where your .Rmd file is open)
\end{itemize}

\textbf{Git/GitHub issues:}

\begin{itemize}
\tightlist
\item
  \textbf{Git pane is empty} → You haven't made any changes yet, or you
  need to save your file first
\item
  \textbf{Can't push} → Make sure you committed first (commit before
  push!)
\item
  \textbf{Authentication error} → Check with your instructor about the
  authentication setup for JupyterHub
\end{itemize}

\textbf{R Markdown issues:}

\begin{itemize}
\tightlist
\item
  \textbf{Output doesn't match what I see in Console} → Click the
  \textbf{Run All Chunks Above} button (down arrow icon) before knitting
\item
  \textbf{My changes aren't showing} → Make sure you saved the file
  (Ctrl+S or Cmd+S)
\end{itemize}

\textbf{General tips:}

\begin{itemize}
\tightlist
\item
  Save your work frequently (Ctrl+S or Cmd+S)
\item
  Knit often to catch errors early
\item
  Read error messages carefully - they often tell you what's wrong
\item
  Don't hesitate to ask for help!
\end{itemize}

\begin{center}\rule{0.5\linewidth}{0.5pt}\end{center}

\section{Exercises}\label{exercises}

\begin{enumerate}
\def\labelenumi{\arabic{enumi}.}
\tightlist
\item
  According to the data dictionary, how many pets are included in this
  dataset?
\end{enumerate}

There are 52,519 pets in the dataset.

\emph{After completing this exercise:}

\begin{enumerate}
\def\labelenumi{\alph{enumi}.}
\tightlist
\item
  Write your answer in the R Markdown document under Exercise 1
\item
  Click \textbf{Knit} to generate the output
\item
  Go to the \textbf{Git pane}, click \textbf{Diff}, then check the boxes
  next to all changed files
\item
  Type commit message: \texttt{"Completed\ Exercise\ 1"}
\item
  Click \textbf{Commit}, then click \textbf{Push}
\item
  Verify your Git pane is cleared (no files listed)
\end{enumerate}

\begin{enumerate}
\def\labelenumi{\arabic{enumi}.}
\setcounter{enumi}{1}
\tightlist
\item
  Again, according to the data dictionary, how many variables do we have
  for each pet? Write your answer in the R Markdown document under
  Exercise 2.
\end{enumerate}

There are 7 variables for each pet.

\emph{After completing this exercise:}

🧶 Knit → ✅ Commit with message \texttt{"Completed\ Exercise\ 2"} → ⬆️
Push

\begin{enumerate}
\def\labelenumi{\arabic{enumi}.}
\setcounter{enumi}{2}
\tightlist
\item
  What are the three most common pet names in Seattle? To do this you
  will need to count the frequencies of each pet name and display the
  results in descending order of frequency so that you can easily see
  the top three most popular names. The following code does exactly
  that.
\end{enumerate}

\begin{marginfigure}
The two lines of code can be read as ``Start with the seattlepets data
frame, and then count the animal\_names, and display the results sorted
in descending order. The `and then' in the previous sentence maps to
\%\textgreater\%, the pipe operator, which takes what comes before it
and plugs it in as the first argument of the function that comes after
it.''
\end{marginfigure}

\begin{Shaded}
\begin{Highlighting}[]
\NormalTok{seattlepets }\SpecialCharTok{\%\textgreater{}\%}
  \FunctionTok{count}\NormalTok{(animal\_name, }\AttributeTok{sort =} \ConstantTok{TRUE}\NormalTok{)}
\end{Highlighting}
\end{Shaded}

\emph{Write your answer in your R Markdown document under Exercise 3. In
this exercise you will not only provide a written answer but also
include some code and output. You should insert the code in the code
chunk provided for you, knit the document to see the output, and then
write your narrative for the answer based on the output of this
function, and knit again to see your narrative, code, and output in the
resulting document.} The three most frequent names are Lucy, Charlie,
and Luna. It is interesting that there are the largest amount for a N?A
answer of 483.

\emph{After completing this exercise:}

🧶 Knit → ✅ Commit with message \texttt{"Completed\ Exercise\ 3"} → ⬆️
Push

Let's also look to see what the most common pet names are for various
species. For this we need to first \texttt{group\_by()} the
\texttt{species}, and then do the same counting we did before.

\begin{marginfigure}
Looks like many of those NAs were cats. Poor unnamed kitties\ldots{}
\end{marginfigure}

\begin{Shaded}
\begin{Highlighting}[]
\NormalTok{seattlepets }\SpecialCharTok{\%\textgreater{}\%} 
  \FunctionTok{group\_by}\NormalTok{(species) }\SpecialCharTok{\%\textgreater{}\%}
  \FunctionTok{count}\NormalTok{(animal\_name, }\AttributeTok{sort =} \ConstantTok{TRUE}\NormalTok{)}
\end{Highlighting}
\end{Shaded}

\begin{verbatim}
## # A tibble: 16,823 x 3
## # Groups:   species [4]
##    species animal_name     n
##    <chr>   <chr>       <int>
##  1 Cat     <NA>          406
##  2 Dog     Lucy          337
##  3 Dog     Charlie       306
##  4 Dog     Bella         249
##  5 Dog     Luna          244
##  6 Dog     Daisy         221
##  7 Dog     Cooper        189
##  8 Dog     Lola          187
##  9 Dog     Max           186
## 10 Dog     Molly         186
## # i 16,813 more rows
\end{verbatim}

But this output isn't exactly what we wanted. We wanted to know the most
common cat and dog names, but there are barely any cats present in this
output! This is because there are more dogs than cats in the dataset
overall. We can confirm this by counting the various species in the
data.

\begin{marginfigure}
6 pigs in the city? Ok\ldots{} But we'll continue with cats and dogs.
\end{marginfigure}

\begin{Shaded}
\begin{Highlighting}[]
\NormalTok{seattlepets }\SpecialCharTok{\%\textgreater{}\%}
  \FunctionTok{count}\NormalTok{(species, }\AttributeTok{sort =} \ConstantTok{TRUE}\NormalTok{)}
\end{Highlighting}
\end{Shaded}

\begin{verbatim}
## # A tibble: 4 x 2
##   species     n
##   <chr>   <int>
## 1 Dog     35181
## 2 Cat     17294
## 3 Goat       38
## 4 Pig         6
\end{verbatim}

\begin{enumerate}
\def\labelenumi{\arabic{enumi}.}
\setcounter{enumi}{3}
\tightlist
\item
  Let's search for the top 5 cat and dog names. To do this, we can use
  the \texttt{slice\_max()} function. The first argument in the function
  is the variable we want to select the highest values of, which is
  \texttt{n}. The second argument is the number of rows to select, which
  is \texttt{n\ =\ 5} for the top 5. It may be a bit confusing that both
  of these are \texttt{n}, but this is because we already have a
  variable called \texttt{n} in the data frame.
\end{enumerate}

\begin{Shaded}
\begin{Highlighting}[]
\NormalTok{seattlepets }\SpecialCharTok{\%\textgreater{}\%} 
  \FunctionTok{group\_by}\NormalTok{(species) }\SpecialCharTok{\%\textgreater{}\%}
  \FunctionTok{count}\NormalTok{(animal\_name, }\AttributeTok{sort =} \ConstantTok{TRUE}\NormalTok{) }\SpecialCharTok{\%\textgreater{}\%} 
  \FunctionTok{arrange}\NormalTok{(species,n) }\SpecialCharTok{\%\textgreater{}\%}
  \FunctionTok{slice\_max}\NormalTok{(n, }\AttributeTok{n =} \DecValTok{5}\NormalTok{)}
\end{Highlighting}
\end{Shaded}

\begin{verbatim}
## # A tibble: 53 x 3
## # Groups:   species [4]
##    species animal_name     n
##    <chr>   <chr>       <int>
##  1 Cat     <NA>          406
##  2 Cat     Luna          111
##  3 Cat     Lucy          102
##  4 Cat     Lily           86
##  5 Cat     Max            83
##  6 Dog     Lucy          337
##  7 Dog     Charlie       306
##  8 Dog     Bella         249
##  9 Dog     Luna          244
## 10 Dog     Daisy         221
## # i 43 more rows
\end{verbatim}

Based on the previous output we can easily identify the most common cat
and dog names in Seattle, but the output is sorted by \texttt{n} (the
frequencies) as opposed to being organized by the \texttt{species}.
Build on the pipeline to arrange the results so that they're arranged by
\texttt{species} first, and then \texttt{n}. This means you will need to
add one more step to the pipeline, and you have two options:
\texttt{arrange(species,\ n)} or \texttt{arrange(n,\ species)}. You
should try both and decide which one organizes the output by species and
then ranks the names in order of frequency for each species.

Which option groups all the cats together and all the dogs together,
with names ranked by frequency within each species?

Arrange(species,n) would do the arrangement as required. \emph{After
completing this exercise:}

🧶 Knit → ✅ Commit with message \texttt{"Completed\ Exercise\ 4} → ⬆️
Push

\begin{enumerate}
\def\labelenumi{\arabic{enumi}.}
\setcounter{enumi}{4}
\tightlist
\item
  The following visualization plots the proportion of dogs with a given
  name versus the proportion of cats with the same name. The 20 most
  common cat and dog names are displayed. The diagonal line on the plot
  is the \(x = y\) line; if a name appeared on this line, the name's
  popularity would be exactly the same for dogs and cats.
\end{enumerate}

\textbf{Tip:} You don't need to understand all the code that creates
this visualization - that will come later in the course. For now, just
look at the plot and answer the questions about what you observe.

\begin{figure*}
\includegraphics[width=0.8\linewidth]{hw-01-pet-names_files/figure-latex/unnamed-chunk-7-1} \end{figure*}

\begin{enumerate}
\def\labelenumi{\alph{enumi}.}
\tightlist
\item
  What names are more common for cats than dogs? The ones above the line
  or the ones below the line?
\item
  Is the relationship between the two variables (proportion of cats with
  a given name and proportion of dogs with a given name) positive or
  negative? What does this mean in context of the data?
\end{enumerate}

\emph{After completing this exercise:}

Unable to complete as there is an error with a missing file -
libMagick++ Error in dyn.load(file, DLLpath = DLLpath, \ldots) : unable
to load shared object `/srv/rlibs/magick/libs/magick.so':
libMagick++-6.Q16.so.8: cannot open shared object file: No such file or
directory

🧶 Knit → ✅ Commit with message \texttt{"Completed\ Exercise\ 5"} → ⬆️
Push

\textbf{To submit to Canvas:}

\begin{enumerate}
\def\labelenumi{\arabic{enumi}.}
\tightlist
\item
  In RStudio, click the \textbf{Knit} dropdown menu (next to the Knit
  button)
\item
  Select \textbf{Knit to tufte\_handout} to generate a PDF
\item
  Download the PDF file from the Files pane
\item
  Upload the PDF to Canvas
\end{enumerate}

\textbf{✓ Final Checkpoint:} Visit your GitHub repo one more time to
confirm all your work is there. We will grade what we see in your repo
on GitHub!



\end{document}
